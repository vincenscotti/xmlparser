\documentclass[italian,a4paper]{article}

\usepackage[T1]{fontenc}
\usepackage[latin9]{inputenc}
\usepackage{babel}
\usepackage{syntax}
\usepackage{minted}
\usepackage{mdframed}
\usepackage{amsmath}

\setminted[C++]{breaklines=true}
\setminted[XML]{breaklines=true}

\mdfsetup{frametitlealignment=\center}

\begin{document}

\title{Realizzazione di un parser per il linguaggio XML}
\author{Vincenzo Scotti M63/693}
\maketitle

\begin{abstract}
	In questo documento viene realizzata una libreria per il parsing del
	linguaggio XML, in particolare adottando un subset della specifica
	ufficiale. Il processo di progettazione e' documentato nella sua
	integrita', dalla fase iniziale di specifica della grammatica fino al
	codice risultante, scritto in C++11. Nel paragrafo finale e' riportato
	un programma che fa uso della suddetta libreria, e vari casi di test.
\end{abstract}

\part{Introduzione}

Si e' deciso di implementare un parser per il linguaggio XML. L'XML e' un
metalinguaggio di markup, ovvero utilizzato per definire dei linguaggi di
markup. Lo standard XML prevede quindi una descrizione sintattica della
struttura del documento, tramite l'utilizzo di \emph{tag} disposti in
maniera gerarchica, con eventualmente attributi a corredo degli stessi. A questo
segue una serie di costrutti per la specifica formale della semantica del
documento, definendo quindi i \emph{tag} ammissibili e come possono essere
innestati tra loro.

Un documento XML e' detto:

\begin{itemize}
	\item \emph{well-formed} se il documento e' sintatticamente corretto;

	\item \emph{valido} se il documento e' semanticamente corretto.
\end{itemize}

Il parser che viene qui realizzato si preoccupa solamente dell'analisi
sintattica di un documento XML. Per questo motivo, per la realizzazione e' stato
tenuto in considerazione solo un subset della specifica completa del linguaggio.

Le soluzioni al parsing di un documento XML possono essere distinte in due
categorie:

\begin{itemize}
	\item i parser \emph{SAX} (Simple API for XML), che prevedono l'analisi
		del documento riga per riga. Questi tipi di parser offrono una
		API \emph{event-driven}, consentendo all'utente di inserire una
		serie di \emph{callback}, che verranno invocate al presentarsi
		di un evento di natura sintattica (ad esempio apertura di un
		tag, chiusura di un tag, etc...);

	\item i parser \emph{DOM} (Document Object Model), che prevedono il
		parsing del documento nella sua integrita'. Per questo motivo il
		documento viene prima caricato in memoria e successivamente
		viene analizzato. Il risultato di questa operazione e' una
		struttura dati gerarchica che rispetta le proprieta' strutturali
		del documento.
\end{itemize}

Si procede quindi all'implementazione di un parser DOM. Il parser effettuera'
l'analisi sintattica del documento e restituira' un errore in caso di documento
non \emph{well-formed}, oppure una struttura dati in caso affermativo, col quale
e' possibile esplorare la struttura del documento.

\part{Grammatica del linguaggio}

Per una specifica completa della grammatica dell'XML si faccia riferimento a
www.w3.org/TR/xml .
Di seguito e' riportato il processo di specifica della grammatica per il solo
subset del linguaggio inerente al lavoro svolto.

La descrizione e' riportata prima in termini informali, utilizzando il
linguaggio naturale, per poi essere formalizzata utilizzando l'eBNF. A parte i
casi descritti di seguito, gli identificatori sono da intendersi come simboli
non terminali.

I simboli terminali sono espressi come:

\begin{itemize}
	\item \emph{\#xNN} dove NN e' un numero, per indicare il valore di un
		byte in esadecimale;

	\item \emph{[ \#xNN to \#xNN ]} per indicare un range di valori
		esadecimali.
\end{itemize}

Nel caso di simboli letterali di tipo stringa, essa sara' riportata interamente
nella regola grammaticale.

\section{Definizione della grammatica}

Un documento XML e' \emph{well-formed} se e' presente un prologo, opzionale,
seguito dalla definizione dell'elemento radice del documento.

\begin{grammar}
	<document> ::= <prologue>? <element>
\end{grammar}

Il prologo e' un tag speciale nel quale viene indicata la versione dello
standard utilizzata nel documento. Allo stato attuale, l'unica versione
consentita e' la \emph{1.0}.

\begin{grammar}
	<prologue> ::= <xml\_decl> <maybe\_space>

	<xml\_decl> ::= "<?xml" <version\_info> <maybe\_space> "?>"

	<version\_info> ::= <space> "version" <equals> ("'1.0'" | "\"1.0\"")

	<equals> ::= <maybe\_space> "=" <maybe\_space>
\end{grammar}

Successivamente segue la definizione dell'elemento radice, che segue le regole
sintattiche di tutti gli altri elementi nel documento. Per questo si introduce
una regola generale che determina la struttura generica di un qualsiasi
elemento.

Un elemento e' composto da un tag di apertura, che ne determina il nome, piu'
una lista opzionale di attributi. L'elemento puo' essere immediatamente chiuso
o puo' possedere un contenuto. Il contenuto dell'elemento puo' essere vuoto,
composto da altri elementi o da una sequenza di caratteri. Ad ogni elemento non
vuoto deve essere associato il rispettivo tag di chiusura.

\begin{grammar}
	<element> ::= "<" <name> (<space> <attribute>)* <maybe\_space> \\
	("/>" | (">" <content> "</" <name> <maybe\_space> ">"))

	<name> ::= (<letter> | "\_" | ":") (<letter> | <digit> | "." | ":" | "-" | "\_")*

	<attribute> ::= <name> <equals> <attribute\_value>

	<attribute\_value> ::= ("\"" (<char> - "\"")* "\"") | ("\'" (<char> - "\'")* "\'")

	<content> ::= (<maybe\_space> <element>)* | <chardata>
\end{grammar}

Infine sono riportati i caratteri terminali, secondo la notazione sopra
riportata.

\begin{grammar}
	<char> ::= [\#x00 to \#x7E]

	<chardata> ::= (<char> - "<")*

	<letter> ::= [\#x41 to \#x7A]

	<digit> ::= [\#x30 to \#x39]

	<space> ::= (\#x20 | \#x9 | \#xD | \#xA)+

	<maybe\_space> ::= <space>?
\end{grammar}

\section{Analisi della grammatica}

Si noti come la regola \syntax{<element>} e' ricorsiva in quanto, durante la sua
espansione, essa puo' essere nuovamente invocata tramite la regola
\syntax{<content>}.

Va inoltre aggiunto che, affinche' un elemento sia valido, il tag di chiusura
deve presentare lo stesso nome del tag di apertura. Questa condizione non e'
espressa nella grammatica eBNF riportata sopra. Infatti un elemento definito
come

\begin{minted}{XML}
<a> ... contenuto ... </b>
\end{minted}

risulta corretto secondo la grammatica specificata, ma non secondo le regole
sintattiche dell'XML.

Questo vincolo sintattico introduce una regola che dipende dal \emph{contesto},
trasformando la sintassi da \emph{context-free} a \emph{context-sensitive}, e
non esprimibile in eBNF. Tuttavia, in fase implementativa, sara' possibile
adottare le soluzioni progettuali per grammatiche \emph{context-free}, con delle
lievi modifiche per tenere conto di questo vincolo aggiuntivo.

Un'altra importante caratteristica da verificare e' se il linguaggio $ \in LL(1)
$, ovvero se da qualsiasi regola di produzione e' possibile sapere con
precisione quale altra regola espandere esaminando solo il token successivo.

Indichiamo formalmente con G una generica grammatica, e con N l'insieme dei
simboli non terminali della grammatica.

\[ G \in LL(1) \iff FIRST(A) \cap FIRST(B) = \emptyset, \forall A, B \in N \]

dove con $ FIRST(A) $ si intende l'insieme dei primi token ottenuti per ogni
espansione di A.

Considerando la grammatica sopra descritta, esaminiamo i simboli
\syntax{<prologue>} e \syntax{<element>}.

\[ FIRST(prologue) = FIRST(xml\_decl) = \, '<' \] 

\[ FIRST(element) = \, '<' \] 

\[ FIRST(prologue) \cap FIRST(element) \neq \emptyset \]

Da questo si puo' concludere che la grammatica $ \notin LL(1) $.

Una spiegazione intuitiva puo' essere data in questo modo: all'inizio
dell'analisi del documento, quando il parser applichera' la regola
\syntax{<document>}, pur supponendo che il prossimo token sia "<", non e'
possibile sapere se esso appartiene alla regola \syntax{<prologue>} o
\syntax{<element>}.

\part{Progettazione del parser}

Un approccio comune per la progettazione di parser per grammatiche
\emph{context-free} e' quello dei parser \emph{ricorsivi discendenti}.

Le regole della grammatica vengono tradotte come funzioni che si richiamano a
vicenda, in maniera ricorsiva. Di fronte a piu' espansioni possibili per una
determinata regola, si puo' procedere in due modi:

\begin{itemize}
	\item con \emph{lookahead}, se la grammatica $ \in LL(k) $.
		Esso consiste nell'analizzare i prossimi $ k $ token per
		determinare univocamente quale espansione applicare;

	\item con \emph{backtracking}, se le varie espansioni sono testate in
		sequenza, selezionando la prima che ha successo.
\end{itemize}

Nel nostro caso e' stato adottato un parser \emph{ricorsivo discendente} con
\emph{backtracking}, con un'opportuna modifica per gestire correttamente la
regola \syntax{<element>}, che rende la grammatica non strettamente
\emph{context-free}.

\section{Implementazione}

La libreria e' stata implementata utilizzando il linguaggio C++, facendo
riferimento allo standard versione 11.

Per una maggiore modularita' si e' fatto uso di un namespace, chiamato
\emph{xml}, per raggruppare le strutture dati e le API.

L'unica funzionalita' offerta dalla libreria, allo stato attuale, e' appunto
quella di effettuare il parsing di un documento, a partire da un iteratore al
carattere iniziale e finale della stringa.

\begin{minted}{C++}
template <typename It>
xml_doc parse(It &s, const It &e);
\end{minted}

La sua implementazione e' discussa in seguito.

\subsection{Strutture dati}

Il documento finale sara' rappresentato nella seguente struttura dati:

\begin{minted}{C++}
class xml_doc
{
public:
        std::string version;
        xml_node *root;

        xml_doc() :
                root{nullptr}
        {   

        }   
};
\end{minted}

La struttura memorizza la versione dell'xml utilizzata nel documento e un
puntatore al nodo radice. Il nodo e' rappresentato utilizzando la seguente
struttura dati:

\begin{minted}{C++}
class xml_node
{
public:
        std::string name;
        std::map<std::string, std::string> attributes;
        std::vector<xml_node *> children;
        std::string value;
};
\end{minted}

Si noti come per ogni nodo viene memorizzato il suo nome, gli eventuali
attributi, gli eventuali nodi figli oppure il valore del suo contenuto.

\subsection{Parsers}

Ogni regola grammaticale viene implementata utilizzando una funzione col
seguente prototipo:

\begin{minted}{C++}
template <typename It>
TipoDiRitorno NomeRegola(It &start, const It &end, ...);
\end{minted}

La funzione prende in ingresso due iteratori, uno che punto al carattere
corrente e l'altro che punta alla fine della sequenza d'ingresso. Il tipo di
ritorno e i parametri aggiuntivi dipenderanno dalla specifica regola che si sta
implementando. Il primo iteratore, sia in caso di successo che di fallimento
della regola, verra' fatto avanzare fino all'ultimo carattere riconosciuto
valido.

In caso di fallimento la funzione lancera' un eccezione di tipo
\emph{parser_error}; in questo modo il \emph{backtracking} viene realizzato
sfruttando il meccanismo di eccezioni del linguaggio. La funzione chiamante
quindi, si preoccupera' di catturare l'eccezione e di provare la regola
successiva, oppure lascera' propagare l'eccezione nel caso non ci siano altre
regole da testare.

\begin{minted}{C++}
class parser_error : public std::exception
{
public:
        parser_error(const std::string &msg);

        virtual const char *what() const noexcept override;

        virtual const std::string error_string() const;

private:
        std::string err_string;
};
\end{minted}

Per una maggiore chiarezza, i parser sono tutti racchiusi nel namespace
\emph{xml::parser}.

Si procede dunque con l'implementazione delle regole della grammatica.

\subsubsection{char}

\begin{grammar}
	<char> ::= [\#x00 to \#x7E]
\end{grammar}

Per realizzare la regola \syntax{<char>} e' stato utilizzato direttamente il tipo
nativo \emph{char} del C++.

\subsubsection{chardata}

\begin{grammar}
	<chardata> ::= (<char> - "<")*
\end{grammar}

La regola chardata si occupa di consumare il buffer in ingresso fino al
carattere "<".

\begin{minted}{C++}
template <typename It> 
std::string chardata(It &s, const It &e) 
{
        auto pos = std::find(s, e, '<');

        s = pos;

        return std::string{s, pos};
}
\end{minted}

\subsubsection{letter, digit}

\begin{grammar}
	<letter> ::= [\#x41 to \#x7A]
\end{grammar}

\begin{grammar}
	<digit> ::= [\#x30 to \#x39]
\end{grammar}

Le regole sono state implementate utilizzando rispettivamente le funzioni
\emph{std::isalpha}, \emph{std::isdigit} e \emph{std::isalnum} fornite dalla
libreria standard del C++.

\subsubsection{space, maybe\_space}

\begin{grammar}
	<space> ::= (\#x20 | \#x9 | \#xD | \#xA)+

	<maybe\_space> ::= <space>?
\end{grammar}

La regola \syntax{<space>} controllera' la presenza di almeno un carattere
spaziatore, e consumera' il buffer fino al primo carattere non spaziatore. La
regola \syntax{<maybe\_space>} e' analoga alla precedente, ma non genera un
fallimento in alcun caso.

\begin{minted}{C++}
template <typename It>
void space(It &s, const It &e)
{
        if (s == e || !std::isspace(*s)) {
                throw parser_error{"space"};
        }

        do {
                s++;
        } while (s != e && std::isspace(*s));
}

template <typename It>
void maybe_space(It &s, const It &e)
{
        try {
                space(s, e);
        } catch (const parser_error &) {

        }
}
\end{minted}

\subsubsection{document}

\begin{grammar}
	<document> ::= <prologue>? <element>
\end{grammar}

Per implementare questa regola e' necessario prima provare a procedere con la
regola \syntax{<prologue>}, e in ogni caso procedere con la regola
\syntax{<element>}.

Trattandosi della regola top-level, il suo valore di ritorno sara' un
\emph{xml\_doc}. La struttura sara' riempita sfruttando i valori di ritorno
delle due regole espanse.

Il meccanismo di gestione delle eccezioni consente di implementare agilmente il
\emph{backtracking}. Infatti, il fallimento della regola \syntax{<prologue>} e'
prontamente rilevato dal blocco catch, che agisce di conseguenza, ovvero in
questo caso ignorando l'errore,, essendo il prologo opzionale.

\begin{minted}{C++}
template <typename It>
xml_doc document(It &s, const It &e)
{
        xml_doc ret;

        try {
                auto curr = s;

                ret = prologue(curr, e);

                s = curr;
        } catch (const parser_error &ex) {
                ret.version = "1.0";
        }

        ret.root = element(s, e);

        return ret;
}
\end{minted}

\subsubsection{prologue}

\begin{grammar}
	<prologue> ::= <xml\_decl> <maybe\_space>
\end{grammar}

Il prologo si occupa di leggere la versione dell'XML utilizzata nel documento, e
di consumare eventuali spazi.

\begin{minted}{C++}
template <typename It>
xml_doc prologue(It &s, const It &e)
{
        xml_doc doc;

        doc.version = xml_decl(s, e);

        maybe_space(s, e);

        return doc;
}
\end{minted}

\subsubsection{xml\_decl}

\begin{grammar}
	<xml\_decl> ::= "<?xml" <version\_info> <maybe\_space> "?>"
\end{grammar}

La dichiarazione XML consiste nell'apertura di un tag, seguita dal numero di
versione e da eventuali spazi prima del tag di chiusura.

\begin{minted}{C++}
template <typename It>
std::string xml_decl(It &s, const It &e)
{
        std::string version;
        auto first_match = "<?xml";
        auto last_match = "?>";

        match_string(s, e, first_match);
        version = version_info(s, e);
        maybe_space(s, e);
        match_string(s, e, last_match);

        return version;
}
\end{minted}

Si noti come i simboli terminali sono consumati utilizzando le funzioni
\emph{match\_string} e \emph{match\_char}, descritte in seguito.

\subsubsection{version\_info}

\begin{grammar}
	<version\_info> ::= <space> "version" <equals> ("'1.0'" | "\"1.0\"")
\end{grammar}

Questa funzione si occupa di consumare l'attributo \emph{version} con l'annesso
valore, che puo' essere pari solo ad \emph{1.0}, come richiesto da specifica.

\begin{minted}{C++}
template <typename It>
std::string version_info(It &s, const It &e)
{
        space(s, e);
        match_string(s, e, "version");
        equals(s, e);

        auto curr = s;

        try {
                match_string(curr, e, "'1.0'");
        } catch (const parser_error &) {
                curr = s;
                match_string(curr, e, "\"1.0\"");
        }

        s = curr;

        return "1.0";
}
\end{minted}

\subsubsection{equals}

\begin{grammar}
	<equals> ::= <maybe\_space> "=" <maybe\_space>
\end{grammar}

Questa regola consuma il simbolo terminale "=", circondato da eventuali spazi.

\begin{minted}{C++}
template <typename It>
void equals(It &s, const It &e)
{
        maybe_space(s, e);
        match_char(s, e, '=');
        maybe_space(s, e);
}
\end{minted}

\subsubsection{element, content}

\begin{grammar}
	<element> ::= "<" <name> (<space> <attribute>)* <maybe\_space> \\
	("/>" | (">" <content> "</" <name> <maybe\_space> ">"))

	<content> ::= (<maybe\_space> <element>)* | <chardata>
\end{grammar}

Questa e' la funzione piu' articolata del parser. Le due regole, per facilitarne
l'implementazione, sono realizzate entrambe in una unica funzione.

La logica di questa funzione si occupa inoltre di controllare la congruenza tra
il nome del tag di apertura e quello di chiusura.

Il valore di ritorno e' di tipo \emph{xml_node *}, ovvero un puntatore ad un
nodo, completo di eventuali figli ed attributi. I nodi figlio sono consumati
richiamando ricorsivamente la funzione stessa.

\begin{minted}{C++}
template <typename It>
xml_node *element(It &s, const It &e)
{
        xml_node *ret = new xml_node;
        bool empty_node;

        try {
                match_char(s, e, '<');
                ret->name = name(s, e);

                while (true) {
                        try {
                                space(s, e);
                                ret->attributes.insert(attribute(s, e));
                        } catch (const parser_error &) {
                                break;
                        }
                }

                try {
                        auto curr = s;

                        match_char(curr, e, '/');
                        empty_node = true;

                        s = curr;
                } catch (const parser_error &) {
                        empty_node = false;
                }

                match_char(s, e, '>');

                if (empty_node) {
                        return ret;
                }

                while (true) {
                        maybe_space(s, e);

                        try {
                                auto curr = s;

                                ret->children.push_back(element(curr, e));

                                s = curr;
                        } catch (const parser_error &) {
                                break;
                        }
                }

                if (ret->children.empty()) {
                        ret->value = chardata(s, e);
                        rtrim(ret->value);
                }

                match_char(s, e, '<');
                match_char(s, e, '/');
                match_string(s, e, ret->name);
                maybe_space(s, e);
                match_char(s, e, '>');
        } catch (const parser_error &) {
                delete ret;
                throw;
        }

        return ret;
}
\end{minted}

\subsubsection{name}

\begin{grammar}
	<name> ::= (<letter> | "\_" | ":") (<letter> | <digit> | "." | ":" | "-" | "\_")*
\end{grammar}

Il nome di un tag puo' iniziare con una lettera o i due caratteri speciali
riportati nella regola. Per i caratteri successivi il nome di un tag puo'
iniziare con una lettera o i due caratteri speciali riportati nella regola.
Successivamente sono consentiti sono consentite anche cifre e ulteriori
caratteri speciali.

\begin{minted}{C++}
template <typename It>
std::string name(It &s, const It &e)
{
        std::string ret;
        char c;

        if (s == e) {
                throw parser_error{"name"};
        }

        c = *s;
        if (!std::isalpha(c) && c != '_' && c != ':') {
                throw parser_error{"name"};
        }

        ret += c;
        s++;

        while (s != e) {
                c = *s;

                if (!std::isalnum(c) && c != '.' && c != ':' && c != '-' && c != '_') {
                        break;
                }

                ret += c;
                s++;
        }

        return ret;
}
\end{minted}

\subsubsection{attribute}

\begin{grammar}
	<attribute> ::= <name> <equals> <attribute\_value>
\end{grammar}

Un attributo e' costituito da una coppia (nome, valore). Per questo motivo il
tipo di ritorno sara' un \emph{std::pair<std::string, std::string>}.

\begin{minted}{C++}
template <typename It>
std::pair<std::string, std::string> attribute(It &s, const It &e)
{
        std::pair<std::string, std::string> ret;

        ret.first = name(s, e);
        equals(s, e);
        ret.second = attribute_value(s, e);

        return ret;
}
\end{minted}

\subsubsection{attribute\_value}

\begin{grammar}
	<attribute\_value> ::= ("\"" (<char> - "\"")* "\"") | ("\'" (<char> - "\'")* "\'")
\end{grammar}

Il valore di un attributo puo' essere racchiuso tra singoli o doppi apici. Ogni
carattere e' consentito (tranne quello terminatore).

\begin{minted}{C++}
template <typename It>
std::string attribute_value(It &s, const It &e)
{
        std::string ret;
        char quote_char = '"';
        char c;

        try {
                match_char(s, e, quote_char);
        } catch (const parser_error &) {
                quote_char = '\'';
                match_char(s, e, quote_char);
        }

        while (s != e) {
                c = *s;

                if (c == quote_char) {
                        s++;
                        break;
                }

                ret += c;
                s++;
        }

        return ret;
}
\end{minted}

\subsection{match\_char, match\_string}

Le funzioni \emph{match\_char} e \emph{match\_string} sono utilizzate per
consumare rispettivamente un carattere o una sequenza di caratteri nella
stringa d'ingresso.

\begin{minted}{C++}
template <typename It> 
void match_char(It &s, const It &e, char c)
{
        if (s == e || *s != c) {
                throw parser_error{std::string{c}};
        }   

        s++;
}

template <typename It> 
void match_string(It &s, const It &e, const std::string &str)
{
        for (char c : str) {
                if (s == e || *s != c) {
                        throw parser_error{str};
                }   

                s++;
        }   
}
\end{minted}

\subsection{API}

La libreria esporta all'utilizzatore una sola funzione chiamata \emph{parse}. In
maniera analoga alle funzioni di parsing, a partire da due iteratori ad inizio e
fine stringa costruisce una struttura \emph{xml\_doc} chiamando la regola
grammaticale di piu' alto livello, ovvero \syntax{<document>}.

\begin{minted}{C++}
template <typename It>
xml_doc parse(It &s, const It &e)
{
	return parser::document(s, e);
}
\end{minted}

\part{Utilizzo del parser}

La libreria sopra progettata e' stata utilizzata per realizzare un semplice
programma. Esso di occupa di leggere in memoria un documento XML da file, e
(verificata la sua correttezza) di ristamparlo utilizzando un formato
differente. Nel caso il documento non sia corretto, viene stampato a video un
messaggio di errore, seguito dalla parte di documento per il quale il parsing e'
fallito.

\begin{minted}{C++}
int main(int argc, char *argv[])
{
        if (argc != 2) {
                return 1;
        }   

        std::ifstream infile(argv[1]);

        if (!infile) {
                std::cerr << "Can't open file " << argv[1] << std::endl;

                return 1;
        }   

        infile >> std::noskipws;

        std::istream_iterator<char> file_it(infile), eof;

        std::string buffer(file_it, eof);
        auto start = buffer.begin();
        const auto end = buffer.end();

        xml::xml_doc doc;

        try {
                doc = xml::parse(start, end);
        } catch (const xml::parser_error &ex) {
                std::cerr << ex.what() << std::endl << std::endl;

                std::cerr << "Chars left: " << std::distance(start, end) << std::endl << std::endl;
                std::for_each(start, end, [] (char c) {
                        std::cerr << c;
                }); 
                std::cerr << std::endl;

                return 1;
        }   

        std::cout << "XML version: " << doc.version << std::endl << std::endl;
        print_node(std::cout, doc.root);
        std::cout << std::endl;

        return 0;
}
\end{minted}

La funzione dedicata alla ristampa del documento e' chiamata \emph{print_node}.
Essa richiede in ingresso un oggetto di tipo \emph{std::ostream} sul quale
effettuare la stampa, un puntatore ad un nodo e la profondita' del nodo (che di
default vale 0), utilizzata per indentare correttamente la struttura a video.

La funzione e' di tipo ricorsivo. Il caso base si verifica quando il nodo in
ingresso e' nullo. Negli altri casi la funzione procede alla stampa del nome e
degli attributi del nodo. Successivamente richiama se stessa su ognuno dei figli
del nodo corrente.

\begin{minted}{C++}
void print_node(std::ostream &os, const xml::xml_node *node, int depth = 0)
{
        if (node == nullptr)
                return;

        for (int i = 0; i < depth; i++) {
                os << " ";
        }

        os << node->name << " [";

        for (auto &p : node->attributes) {
                os << p.first << "=" << p.second << ", ";
        }   

        os << "]";

        if (!node->value.empty()) {
                os << " = " << node->value;
        }   

        os << std::endl;

        for (auto &c : node->children) {
                print_node(os, c, depth + 1); 
        }   
}
\end{minted}

La complessita' di questo algoritmo e' uguale a quella di una \emph{Depth-first
search} su un albero con N nodi, ovvero pari a $ \Theta(N) $.

\subsection{Esempi di parsing completo}

\subsubsection{Esempio 1}

\begin{mdframed}[frametitle=Input]
\begin{minted}{XML}
<breakfast_menu>
        <food>
                <name>Belgian Waffles</name>
                <price>$5.95</price>
                <description>
                Two of our famous Belgian Waffles
                </description>
                <calories>650</calories>
        </food>
        <food>
                <name>Strawberry Belgian Waffles</name>
                <price>$7.95</price>
                <description>
                Light Belgian waffles covered with milk
                </description>
                <calories>900</calories>
        </food>
        <food>
                <name>French Toast</name>
                <price>$4.50</price>
                <description>
                Thick slices made from our homemade bread
                </description>
                <calories>600</calories>
        </food>
        <food>
                <name>Homestyle Breakfast</name>
                <price>$6.95</price>
                <description>
                Two eggs, bacon or sausage, toast
                </description>
                <calories>950</calories>
        </food>
</breakfast_menu>
\end{minted}
\end{mdframed}

\begin{mdframed}[frametitle=Output]
\begin{verbatim}
XML version: 1.0

breakfast_menu []
 food []
  name [] = Belgian Waffles
  price [] = $5.95
  description [] = Two of our famous Belgian Waffles
  calories [] = 650
 food []
  name [] = Strawberry Belgian Waffles
  price [] = $7.95
  description [] = Light Belgian waffles covered with milk
  calories [] = 900
 food []
  name [] = French Toast
  price [] = $4.50
  description [] = Thick slices made from our homemade bread
  calories [] = 600
 food []
  name [] = Homestyle Breakfast
  price [] = $6.95
  description [] = Two eggs, bacon or sausage, toast
  calories [] = 950
\end{verbatim}
\end{mdframed}

\subsubsection{Esempio 2}

\begin{mdframed}[frametitle=Input]
\begin{minted}{XML}
<note priority="high">
        <to>Tove</to>
        <from>Jani</from>
        <heading>Reminder</heading>
        <body>Don't forget me this weekend!</body>
</note>
\end{minted}
\end{mdframed}

\begin{mdframed}[frametitle=Output]
\begin{verbatim}
XML version: 1.0

note [priority=high, ]
 to [] = Tove
 from [] = Jani
 heading [] = Reminder
 body [] = Don't forget me this weekend!
\end{verbatim}
\end{mdframed}

\subsubsection{Esempio 3}

\begin{mdframed}[frametitle=Input]
\begin{minted}{XML}
<CATALOG>
        <PLANT>
                <COMMON>Bloodroot</COMMON>
                <BOTANICAL>Sanguinaria canadensis</BOTANICAL>
                <ZONE>4</ZONE>
                <LIGHT>Mostly Shady</LIGHT>
                <PRICE>$2.44</PRICE>
                <AVAILABILITY>031599</AVAILABILITY>
        </PLANT>
        <PLANT>
                <COMMON>Columbine</COMMON>
                <BOTANICAL>Aquilegia canadensis</BOTANICAL>
                <ZONE>3</ZONE>
                <LIGHT>Mostly Shady</LIGHT>
                <PRICE>$9.37</PRICE>
                <AVAILABILITY>030699</AVAILABILITY>
        </PLANT>
        <PLANT>
                <COMMON>Marsh Marigold</COMMON>
                <BOTANICAL>Caltha palustris</BOTANICAL>
                <ZONE>4</ZONE>
                <LIGHT>Mostly Sunny</LIGHT>
                <PRICE>$6.81</PRICE>
                <AVAILABILITY>051799</AVAILABILITY>
        </PLANT>
</CATALOG>
\end{minted}
\end{mdframed}

\begin{mdframed}[frametitle=Output]
\begin{verbatim}
XML version: 1.0

CATALOG []
 PLANT []
  COMMON [] = Bloodroot
  BOTANICAL [] = Sanguinaria canadensis
  ZONE [] = 4
  LIGHT [] = Mostly Shady
  PRICE [] = $2.44
  AVAILABILITY [] = 031599
 PLANT []
  COMMON [] = Columbine
  BOTANICAL [] = Aquilegia canadensis
  ZONE [] = 3
  LIGHT [] = Mostly Shady
  PRICE [] = $9.37
  AVAILABILITY [] = 030699
 PLANT []
  COMMON [] = Marsh Marigold
  BOTANICAL [] = Caltha palustris
  ZONE [] = 4
  LIGHT [] = Mostly Sunny
  PRICE [] = $6.81
  AVAILABILITY [] = 051799
\end{verbatim}
\end{mdframed}

\subsection{Esempi di parsing incompleto}

\subsubsection{Esempio 1}

Al nome del tag \emph{note} viene aggiunto il carattere \emph{!}.

\begin{mdframed}[frametitle=Input]
\begin{minted}{XML}
<!note>
  <to>Tove</to>
  <from>Jani</Ffrom>
  <heading>Reminder</heading>
  <body>Don't forget me this weekend!</body>
</note>
\end{minted}
\end{mdframed}

\begin{mdframed}[frametitle=Output]
\begin{verbatim}
token not found: name

Chars left: 123

<!note>
        <to>Tove</to>
        <from>Jani</Ffrom>
        <heading>Reminder</heading>
        <body>Don't forget me this weekend!</body>
</note>
\end{verbatim}
\end{mdframed}

\subsubsection{Esempio 2}

La versione XML indicata e' diversa dalla \emph{1.0}.

\begin{mdframed}[frametitle=Input]
\begin{minted}{XML}
<?xml version="2.0">
<!note>
	<to>Tove</to>
	<from>Jani</Ffrom>
	<heading>Reminder</heading>
	<body>Don't forget me this weekend!</body>
</note>
\end{minted}
\end{mdframed}

\begin{mdframed}[frametitle=Output]
\begin{verbatim}
token not found: name

Chars left: 295

?xml version="2.0">
<!note>                                                   
        <to>Tove</to>                                     
        <from>Jani</Ffrom>                                
        <heading>Reminder</heading>                       
        <body>Don't forget me this weekend!</body>        
</note>
\end{verbatim}
\end{mdframed}

Si noti come l'errore segnalato sembrerebbe riguardare la regola \syntax{<name>},
anche se la regola violata e' in realta' \syntax{<version\_info>}. Questo e'
dovuto al fatto che, una volta fallita la regola \syntax{<version\_info>}, il
fallimento si propaga fino alla regola \syntax{<prologue>}. A questo punto il
parser prova ad invocare la regola \syntax{<element>}, che anch'essa fallisce in
quanto un punto esclamativo non e' valido come primo carattere del nome di un
elemento.

\subsubsection{Esempio 3}

Il documento e' terminato prematuramente.

\begin{mdframed}[frametitle=Input]
\begin{minted}{XML}
<?xml version="1.0" ?>
<note>    
        <to>Tove</to>
\end{minted}
\end{mdframed}

\begin{mdframed}[frametitle=Output]
\begin{verbatim}
token not found: <

Chars left: 0
\end{verbatim}
\end{mdframed}

\end{document}
